\begin{landscape}
\begin{longtable}{>{\textsf\bgroup}p{2.5cm}<{\egroup} >{\textsf\bgroup}p{16cm}<{\egroup}} % defining the columns - these must match the widths defined for the mini pages down below!
\caption{The NSL complex is composed of evolutionarily conserved proteins that are largely uncharacterized in \textit{D. melanogaster} (n.a. – no information available).} \\
%%%%%%%%%%%%%%%
% table title
\textbf{Protein} & \textbf{Known biological functions}
\tabularnewline \hline
\endfirsthead % indicates that the lines above appear as head of the table on the first page
\multicolumn{2}{c}%
{\tablename\ \thetable\ -- \textit{Continued from previous page}} \\
\textbf{Protein} & \textbf{Known biological functions}
\endhead % Line(s) to appear at top of every page (except first)
\hline \multicolumn{2}{r}{\textit{Continued on next page}} \\
\endfoot % Last line(s) to appear at the bottom of every page (except last)
\endlastfoot
%%%%%%%%%%%%%%%%%%%
%%% let's start the table content; each column (often) gets its own minipage which enables itemized lists etc.
%%%%%%%%%%%%%%%%%%%
\begin{minipage}{2.5cm} % 1st column
				%\vskip 6pt
					\textbf{NSL1} \\
					hMSL1vs, Kansl1
				%\vskip 4pt
			\end{minipage}
				& \begin{minipage}{16cm} % 2nd column
					%	\vskip 6pt
						\begin{itemize}[noitemsep]
							\item interacts with MOF via PEHE domain [7, 18] \\
							\item Kansl1 conveys acetylation of H4K16 and p53 through MOF interaction [Li, 2009]
							\item interacts with WDR5
							\item \textit{Kansl1} haploinsufficiency causes core phenotype of 17q21.31 microdeletion syndrome [24, 25]
						\end{itemize}				
						\vskip 4pt
					\end{minipage}
\tabularnewline \hline  % new row
%-----------------------
\begin{minipage}{2.5cm}
				\vskip 6pt
					\textbf{NSL2} \\
					Kansl2
				\vskip 4pt
			\end{minipage}
						& n.a
\tabularnewline \hline
%-----------------------
\begin{minipage}{2.5cm}
				\vskip 6pt
					\textbf{NSL3} \\
					KIAA1310 \\
					Kansl3
				\vskip 4pt
			\end{minipage}
					& \begin{minipage}{16cm}
						\vskip 6pt
						\begin{itemize}[noitemsep]
							\item can activate gene expression
						\end{itemize}				
						\vskip 4pt
					\end{minipage}
\tabularnewline \hline
%-----------------------
\begin{minipage}{2.5cm}
				\vskip 6pt
					\textbf{MCRS2} \\
					MSP58 \\
					Mcrs1
				\vskip 4pt
			\end{minipage}
					& \begin{minipage}{16cm}
						\vskip 6pt
						\begin{itemize}[noitemsep]
							\item interaction with Pol II machinery [20]
							\item mammalian Msp58 suggested in nucleolar protein trafficking and rRNA gene transcription enhancement [26]
							\end{itemize}				
						\vskip 4pt
					\end{minipage}
\tabularnewline \hline
%-----------------------
\begin{minipage}{2.5cm}
				\vskip 6pt
					\textbf{MBD-R2} \\
					Phf20
				\vskip 4pt
			\end{minipage}
				& \begin{minipage}{16cm}
						\vskip 6pt
						\begin{itemize}[noitemsep]
							\item general transcription activator [21]
							\item binds histone (di)methyl lysine [27]
							\item \textit{Phf20} ko mice die perinatally with skeletal and hematopoietic abnormalities [28]
							\item antigen against PHF20 is frequently expressed in glioma
						\end{itemize}				
						\vskip 4pt
					\end{minipage}
\tabularnewline \hline
%-----------------------
\begin{minipage}{2.5cm}
				\vskip 6pt
					\textbf{WDS} \\
					Wdr5 \\
				\vskip 4pt
			\end{minipage}
					& \begin{minipage}{16cm}
						\vskip 6pt
						\begin{itemize}[noitemsep]
							\item WD repeats suggest multiple protein interactions [30]
							\item part of several mammalian methyl transferase complexes (MLL, ATAC [23,31,32])
							\item interacts with SRY yo activate Sox9 expression in human prostate cell line [33]
							\item Wdr5 expression correlates with undifferentiated state of embryonic stem cells [34]
						\end{itemize}				
						\vskip 4pt
					\end{minipage}
\tabularnewline \hline %\end{table}
%-----------------------
\label{tab:NSL}
\end{longtable}
\end{landscape}