\noindent\begin{minipage}{\textwidth} % to avoid pagebreaks despite longtable environment
%\begin{landscape}
\begin{singlespacing}
\begin{small}
\vspace*{-2em}
%\setlength{\abovecaptionskip}{-10pt}
\begin{longtable}{>{\textsf\bgroup\raggedleft\arraybackslash}p{3cm}<{\egroup} >{\textsf\bgroup}p{5cm}<{\egroup} >{\textsf\bgroup}p{6cm}<{\egroup}} % defining the columns - these must match the widths defined for the mini pages down below!
\caption[Biases and artifacts of ChIP-seq data.]{\textsf{Biases and artifacts of ChIP-seq data. Given a rigorously tested antibody, ChIP-seq still suffers from additional technical problems that are due to the sequencing process as well as bioinformatic hurdles.}} \\ % the \\ is important! see http://tex.stackexchange.com/questions/103698/extra-alignment-tab-with-longtable
%%%%%%%%%%%%%%%
% table title
\textbf{Problem} & \textbf{Reasons} & \textbf{Solutions}
\tabularnewline
%-----------------------
\toprule
\begin{minipage}{3cm}
				%\vskip 6pt
					\textbf{Chromatin context} and \textbf{transcription}
				%\vskip 4pt
			\end{minipage}
			&	\begin{minipage}{5cm}
				%\vskip 6pt
				\begin{itemize}[noitemsep,leftmargin=*]
					\item euchromatic chromatin is more easily fragmented
					\item formaldehyde fixation does not capture short-lived protein-DNA interactions \citep{Gavrilov2014} and pref\-er\-en\-tially cross\-links proteins with each other \citep{Neill2003}
					\item chromatin extraction may unspecifically enrich for highly active genes \citep{Teytelman2013}
				
				\end{itemize}
					%\vskip 4pt
			\end{minipage}
			& \begin{minipage}{6cm}
				\vskip 6pt
					\begin{itemize}[noitemsep,leftmargin=*]
						\item cell-type- and condition-specific input controls \citep{Vega2009, Landt2012}
						\item if possible, avoid crosslinking \citep{Neill2003, Kasinathan2014}
						\item optimized chromatin extraction including extensive de-crosslinking, RNase and Proteinase treatments \cite{Waldminghaus2010}
						\item to identify hyper-ChIPable regions, ChIP against a non-endogenous protein was suggested \citep{Teytelman2013}
						\item immunoprecipitation with immunoglobulin G (IgG, mock IP) \citep{Landt2012,Marinov2014}
						
					\end{itemize}	
				\vskip 4pt
			\end{minipage}
\tabularnewline  \hline 
%---------------------------
\begin{minipage}{3cm}
				%\vskip 6pt
					\textbf{Sequencing errors\\and errors in base calling}
				%\vskip 4pt
			\end{minipage}
			& \begin{minipage}{5cm}
				\vskip 6pt
				\begin{itemize}[noitemsep,leftmargin=*]
					\item imperfect sequencing chemistry and signal detection
					\item loss of synchronized base-incorporation into the single molecules within one cluster of clonally amplified DNA frag\-ments (phasing and pre-phasing) (see \tref{tab:sequencing})
					\item signal intensity decay
				\end{itemize}
					\vskip 4pt
			\end{minipage}
				& \begin{minipage}{6cm}
				%\vskip 6pt
					\begin{itemize}[noitemsep,leftmargin=*]
					\item improvement of the sequencing chemistry and detection
					\item optimized software for base calling \citep{Ledergerber2011}
					\item computational removal of bases with low base calling scores \citep{Minoche2011}
					\end{itemize}
				%\vskip 4pt
				\end{minipage}
\tabularnewline  \hline 
%-----------------------
\begin{minipage}{3cm}
				%\vskip 6pt
					\textbf{GC bias} and\\
					\textbf{duplicate reads}
				%\vskip 4pt
			\end{minipage}
			&	\begin{minipage}{5cm}
				%\vskip 6pt
					\begin{itemize}[noitemsep,leftmargin=*]
						\item GC-rich regions are preferably amplified by PCR
						\item small fragments are preferably hybridized to the flow cell
						\item low number of founder DNA frag\-ments
					\end{itemize}
				%\vskip 4pt
			\end{minipage}
			& \begin{minipage}{6cm}
				\vskip 6pt
					\begin{itemize}[noitemsep,leftmargin=*]
						\item optimizing cross-linking, sonication, and the ChIP protocol to ensure that the majority of the genome is present in the sample
						\item limiting PCR cycles during library preparation to a minimum
						\item computational correction for GC content \citep{Cheung2011, Benjamini2012} and elimination of reads from identical DNA fragments
					\end{itemize}	
				\vskip 4pt
			\end{minipage}
\tabularnewline  \hline 
%-----------------------------
\begin{minipage}{3cm}
				%%\vskip 6pt
					\textbf{Copy number\\variations and\\mappability}
				%\vskip 4pt
			\end{minipage}
			&
			\begin{minipage}{5cm}
				%\vskip 6pt
				\begin{itemize}[noitemsep,leftmargin=*]
					\item incomplete genome assemblies
					\item strain-specific differences to the reference assembly may lead to misrepresentation of individual loci
					\item repetitiveness of genomes and shortness of sequencing reads hinder unique read alignment
				\end{itemize}
					%\vskip 4pt
			\end{minipage}
			& \begin{minipage}{6cm}
				\vskip 6pt
					\begin{itemize}[noitemsep,leftmargin=*]
										\item increased sequencing depth and control (non-ChIP) sample aid the computational identification of problematic loci \citep{Bailey2013, Chen2012, Jung2014, Landt2012, Kidder2011}
										\item longer sequencing reads
										\item paired-end sequencing \citep{Chen2012, Bailey2013}
										\item exclusion of blacklisted regions that are known to attract artificially high read numbers \citep{Carroll2014, Blacklists}
										\item computational correction for mappability \citep{Cheung2011}
										\item considering the \textit{effective} genome size \citep{Zhang2008}
					\end{itemize}	
				\vskip 4pt
			\end{minipage}
\tabularnewline \bottomrule
%-----------------------------
\label{tab:biases}
\end{longtable}
\end{small}
\end{singlespacing}
%\end{landscape}
\end{minipage}
