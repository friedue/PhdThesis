% TABLE: x chromosomes in flies and mammals
\begin{longtable}{>{\textsf\bgroup}p{7.5cm}<{\egroup} >{\textsf\bgroup}p{7.5cm}<{\egroup}} % defining the columns - these must match the widths defined for the mini pages down below!
\caption{Characteristics of the X chromosome in mammals and flies.} \\
%%%%%%%%%%%%%%%
% table title
\textbf{Mammalian X} & \textbf{Fly X}
\tabularnewline \hline
\endfirsthead % indicates that the lines above appear as head of the table on the first page
\multicolumn{2}{c}%
{\tablename\ \thetable\ -- \textit{Continued from previous page}} \\
\textbf{Mammalian X} & \textbf{Fly X}
\endhead % Line(s) to appear at top of every page (except first)
\hline \multicolumn{2}{r}{\textit{Continued on next page}} \\
\endfoot % Last line(s) to appear at the bottom of every page (except last)
\endlastfoot
%%%%%%%%%%%%
%%% let's start the table content; each column (often) gets its own minipage which enables itemized lists etc.
%%%%%%%%%%%%
ca. 5\% of all genes \cite{Ross2005}
	& 2,224 (ca. 15\% of all genes)
\tabularnewline \hline  % new row
overrepresentation of narrowly expressed genes \cite{Mikhaylova2011} & overrepresentation of broadly expressed genes \cite{Lercher2003}
\tabularnewline \hline
many brain-related genes \cite{Yang2006} & many head-related genes \cite{Chang2011}
\tabularnewline \hline
overrepresentation of prostate-specific genes \cite{Lercher2003, Meisel2012} & paucity of genes for accessory gland proteins \cite{Mueller2005}, overrepresentation of female-specific genes \cite{Meisel2012}
\tabularnewline \hline
\label{tab:X}
\end{longtable}

