\vspace{-2em}
%\begin{minipage}{\textwidth} % to avoid pagebreaks despite longtable environment
\begin{landscape}
\begin{singlespacing}
\begin{small}
\begin{longtable}{>{\textsf\bgroup\raggedleft\arraybackslash}p{2cm}<{\egroup} >{\textsf\bgroup}p{6.5cm}<{\egroup} >{\textsf\bgroup}p{6.1cm}<{\egroup}>{\textsf\bgroup}p{6.7cm}<{\egroup}} % defining the columns - these must match the widths defined for the mini pages down below!
\caption[Quality metrics of ChIP-seq experiments.]{\textsf{Quality metrics of ChIP-seq experiments. The sequencing is strongly influenced by the success of the sample and library preparation (see \tref{tab:biases}). The metrics shown here should guide the data processing and raise awareness for possible problems during downstream analyses. The failure of a ChIP-seq experiment to meet the recommended criteria does not immediately imply that no biologically meaningful information could be derived from it as ChIP-seq experiments with very few or very diffuse binding sites will always perform worse for genome-wide measures of signal-to-noise-ratios.}} \\ % the \\ is important! see http://tex.stackexchange.com/questions/103698/extra-alignment-tab-with-longtable
%%%%%%%%%%%%%%%
% table title
\textbf{Property} & \textbf{Rationale} & \textbf{Recommendation} & \textbf{Limitation}
\tabularnewline \hline
\endfirsthead % indicates that the lines above appear as head of the table on the first page
%%%%%%%%%%%%%%%%%%%%
\multicolumn{4}{c}%
{\tablename\ \thetable\ -- \textit{Continued from previous page}} \\[1ex]
\textbf{Metric} & \textbf{Rationale} & \textbf{Recommendation} & \textbf{Limitation}
\tabularnewline \toprule %\tabularnewline [1ex]
\endhead % Line(s) to appear at top of every page (except first)
%%%%%%%%%%%%%%%%%%%%%%%
\multicolumn{4}{r}{\textit{Continued on next page}} \\
\endfoot % Last line(s) to appear at the bottom of every page (except last)
\endlastfoot
\tabularnewline
%-----------------------
%\toprule
\multicolumn{4}{c}{\normalsize{\textsc{Library and sequencing quality}}}
\tabularnewline \bottomrule
%--------------------------
\begin{minipage}{2cm}
				%\vskip 6pt
			\raggedright Sequencing depth
				%\vskip 4pt
\end{minipage}
			&	\begin{minipage}{6.5cm}
				%\vskip 6pt
				 measure of how many times the genome is covered with DNA reads (on average):\\
				\raggedright $Coverage = reads \times f\!ragment\;length^{}/genome\;size$
			  	%\vskip 4pt
			\end{minipage}
			& \begin{minipage}{6.1cm}
				\vskip 6pt
					\begin{itemize}[noitemsep,leftmargin=*]
				\item majority of ChIP-seq studies: coverage between 1-5 \citep{Sims2014}
				\item should always be complemented by the number of bases with zero coverage
			\end{itemize}
				\vskip 4pt
			\end{minipage}
			& \begin{minipage}{6.7cm}
			average coverage does not take gaps and uneven read distributions into account \citep{Sims2014}
			\end{minipage}
\tabularnewline  \midrule
%---------------------------
\begin{minipage}{2cm}
				%\vskip 6pt
			\raggedright Over\-rep\-re\-sen\-ted sequences
				%\vskip 4pt
\end{minipage}
			&	\begin{minipage}{6.5cm}
				%\vskip 6pt
				useful for the identification of:
				\begin{itemize}[noitemsep,leftmargin=*]
				  \item adapter and foreign DNA contamination
					\item suggestion of GC bias
			  \end{itemize}
								%\vskip 4pt
			\end{minipage}
			& \begin{minipage}{6.1cm}
				\vskip 6pt
					\begin{itemize}[noitemsep,leftmargin=*]
						\item can be assessed with FASTQC \citep{FASTQC}
						\item contamination should be removed, e.g. with Trimmomatic \cite{Bolger2014}
						\item GC bias should be re-assessed after read alignment and filtering, e.g. with deepTools \citep{Ramirez2014}
					\end{itemize}
				\vskip 4pt
			\end{minipage}
			& \begin{minipage}{6.7cm}
					\begin{itemize}[noitemsep,leftmargin=*]
						\item FASTQC will always work on a sample of sequences, not the full data set
						\item contamination removal is computationally intensive
			\end{itemize}
			\end{minipage}
\tabularnewline  \midrule
%---------------------------
\begin{minipage}{2cm}
				%\vskip 6pt
				\raggedright PCR bot\-tle\-neck co\-ef\-fi\-cient (PBC)
				%\vskip 4pt
\end{minipage}
			&	\begin{minipage}{6.5cm}
				%\vskip 6pt
				%\begin{itemize}[noitemsep,leftmargin=*]
				suggested metric for sequencing library complexity: ratio of genomic locations with \textit{only one} aligned read to the number of loci with \textit{at least} one aligned read \citep{ENCODEMetrics}
				%\end{itemize}
					%\vskip 4pt
			\end{minipage}
			& \begin{minipage}{6.1cm}
				%\vskip 6pt
					\begin{itemize}[noitemsep,leftmargin=*]
						\item can be calculated with ENCODE tools \citep{ENCODETools}
						\item 0-0.5: severe bias, 0.9-1.0: no bias
					\end{itemize}
				%\vskip 4pt
			\end{minipage}
			& \begin{minipage}{6.7cm}
                \vskip 6pt
					\begin{itemize}[noitemsep,leftmargin=*]
						\item the more deeply a sample is sequenced, the more likely it is that duplicated reads could correspond to different DNA fragments \citep{Zhang2008, Landt2012, Yan2014}
						\item duplication ratios might be over-estimated for ChIP-seq samples with very localized and very strong enrichments \citep{Chen2012} 
			\end{itemize}
            \vskip 4pt
			\end{minipage}
\tabularnewline  \midrule
%---------------------------
\begin{minipage}{2cm}
				%\vskip 6pt
				GC bias
				%\vskip 4pt
\end{minipage}
			&	\begin{minipage}{6.5cm}
				%\vskip 6pt
				%\begin{itemize}[noitemsep,leftmargin=*]
					%\item
					after read alignment, the observed read numbers per GC content can be compared to the expected values based on the reference genome sequence (\aref{SuppPub_deepTools})
				%\end{itemize}
									%\vskip 4pt
			\end{minipage}
			& \begin{minipage}{6.1cm}
				\vskip 6pt
					\begin{itemize}[noitemsep,leftmargin=*]
						\item the majority of the genome should not show significant deviation from $observed^{}/expected = 1$ \citep{bamFingerprint}
						\item deepTools can be used to computationally adjust the ratio to 1 (\aref{SuppPub_deepTools})
					\end{itemize}
				\vskip 4pt
			\end{minipage}
			& \begin{minipage}{6.7cm}
					\begin{itemize}[noitemsep,leftmargin=*]
						\item ChIP-seq of factors binding GC-rich regions
						\item correction for lack of AT-rich sequences leads to artificial introduction of duplicate reads \citep{bamFingerprint}
			\end{itemize}
			\end{minipage}
\tabularnewline 
%---------------------------
\multicolumn{4}{c}{\normalsize{\textsc{Reproducibility}}}
\tabularnewline \bottomrule
%----------------------------
\begin{minipage}{2cm}
				%\vskip 6pt
				\raggedright Correlation of read cov\-er\-ages
				%\vskip 4pt
\end{minipage}
			&	\begin{minipage}{6.5cm}
				\vskip 6pt
				\begin{itemize}[noitemsep,leftmargin=*]
					\item two replicates are recommended for each ChIP-seq and input experiment \citep{Rozowsky2009, Landt2012}
					\item corresponding replicates should show similar (highly correlating) read coverage distributions \citep{Bardet2012}
				\end{itemize}
				\vskip 4pt
			\end{minipage}
			& \begin{minipage}{6.1cm}
				%\vskip 6pt
					\begin{itemize}[noitemsep,leftmargin=*]
						\item correlations can be visually represented with multi-dimensional scaling plots \citep{Planet2012} or clustered heatmaps (\aref{SuppPub_deepTools})
						\item replicates should cluster together
					\end{itemize}
				%\vskip 4pt
			\end{minipage}
			& \begin{minipage}{6.7cm}
					Pearson correlation is sensitive to outliers which might inflate the correlation coefficient \citep{Bailey2013}
			\end{minipage}
\tabularnewline  \midrule %\pagebreak
%----------------------------
\begin{minipage}{2cm}
				%\vskip 6pt
				\raggedright Ir\-re\-pro\-duci\-bi\-lity Discovery Rate (IDR)
				%\vskip 4pt
\end{minipage}
			&	\begin{minipage}{6.5cm}
				%\vskip 6pt
				\begin{itemize}[noitemsep,leftmargin=*]
					\item metric for consistency between two replicates \citep{Li2011}
					\item based on the comparison of the ranks of peak regions identified in both replicates
				\end{itemize}
									%\vskip 4pt
			\end{minipage}
			& \begin{minipage}{6.1cm}
				%\vskip 6pt
					\begin{itemize}[noitemsep,leftmargin=*]
						\item all measures output by the IDR package should be within a factor of two \citep{Landt2012}
							\end{itemize}
				%\vskip 4pt
			\end{minipage}
			& \begin{minipage}{6.7cm}
\vskip 6pt
					\begin{itemize}[noitemsep,leftmargin=*]
					\item strongly depends on the results of the peak calling step (regions as well as significance measures); requires very relaxed peak calling parameters \citep{Li2011, Bailey2013}
					\item results are dominated by the weakest replicate \citep{Landt2012}
						\item not recommended for broad enrichments \citep{KundajeIDR}
			\end{itemize}
\vskip 4pt
			\end{minipage}
\tabularnewline  \midrule
%---------------------------
\multicolumn{4}{c}{\normalsize\textsc{Success of the immunoprecipitation}}
\tabularnewline \bottomrule
%----------------------------
\begin{minipage}{2cm}
				\vskip 6pt
				\raggedright Standardized Standard Deviation (SSD)
				\vskip 4pt
\end{minipage}
			&	\begin{minipage}{6.5cm}
				\vskip 6pt
			describes the variation in signal depth across the genome \citep{Carroll2014}:
			$ssd = 1000 \times sd/\sqrt{n}$ \citep{Planet2012}
					\vskip 4pt
			\end{minipage}
			& \begin{minipage}{6.1cm}
				\vskip 6pt
				ChIP and input samples should show different SSD values \citep{Planet2012}
				\vskip 4pt
			\end{minipage}
			& \begin{minipage}{6.7cm}
\vskip 6pt
					sensitive to outlier regions with artificially high coverage in both sample types \citep{Carroll2014}
\vskip 4pt
			\end{minipage}
\tabularnewline  \midrule
%---------------------------
\begin{minipage}{2cm}
				%\vskip 6pt
				\raggedright Cumulative percentages of read counts
				%\vskip 4pt
\end{minipage}
			&	\begin{minipage}{6.5cm}
				%\vskip 6pt
				\begin{itemize}[noitemsep,leftmargin=*]
					\item an ideal, strong ChIP-seq samples should have relatively few regions that contain a large fraction of DNA reads (= enrichments) \citep{Diaz2012a}
				\end{itemize}
									%\vskip 4pt
			\end{minipage}
			& \begin{minipage}{6.1cm}
				\vskip 6pt
ideally, input and ChIP-seq sample should show clearly different cumulative percentages \citep{Diaz2012, Diaz2012a, bamFingerprint} (\aref{SuppPub_deepTools})
				\vskip 4pt
			\end{minipage}
			& \begin{minipage}{6.7cm}
ChIP-seq samples with broad, domain-like enrichments are not well represented
			\end{minipage}
\tabularnewline  \midrule
%---------------------------
\begin{minipage}{2cm}
				%\vskip 6pt
				\raggedright Normalized and relative strand cross-cor\-re\-lation (NSC, RSC)
				%\vskip 4pt
\end{minipage}
			&	\begin{minipage}{6.5cm}
				\vskip 6pt
				\begin{itemize}[noitemsep,leftmargin=*]
				  \item DNA reads are counted separately for forward and reverse strand, then the cross-correlation (cc) is calculated for incremental distances between the strand-specific coverages \citep{Landt2012, Bailey2013, ENCODEMetrics}
					\item $nsc = cc_{f\!ragmentLength}^{}/cc_{background}$ 
					\item $rsc = cc_{f\!ragmentLength}^{}/cc_{readLength}$
				\end{itemize}
				\vskip 4pt
			\end{minipage}
			& \begin{minipage}{6.1cm}
				%\vskip 6pt
				NSC $\geq$ 1.05 and RSC $\geq$ 0.8 \citep{Landt2012}
				%\vskip 4pt
			\end{minipage}
			& \begin{minipage}{6.7cm}
			broad enrichments and factors with few binding sites will meet the suggested threshold \citep{Landt2012, Bailey2013}
			\end{minipage}
\tabularnewline  \midrule
%---------------------------
\begin{minipage}{2cm}
				%\vskip 6pt
				\raggedright Fraction of reads in peaks (FRiP)
				%\vskip 4pt
\end{minipage}
			&	\begin{minipage}{6.5cm}
				%\vskip 6pt
				simple proxy for the success of an IP:\\
				$f\!rip = reads\;in\;peaks^{}/total\;reads$ \citep{Landt2012}
													%\vskip 4pt
			\end{minipage}
			& \begin{minipage}{6.1cm}
				%\vskip 6pt
				FRiP $\geq$ 1\%
				%\vskip 4pt
			\end{minipage}
			& \begin{minipage}{6.7cm}
\vskip 6pt
					\begin{itemize}[noitemsep,leftmargin=*]
						\item requires peak calling
						\item depends strongly on the nature of the ChIP-seq signal (width, strength, number of peak regions) and should therefore not be used to compare ChIP-seq experiments for different proteins of interest \citep{Landt2012}
\vskip 6pt
					\end{itemize}
			\end{minipage}
\tabularnewline \bottomrule
%-----------------------------
\label{tab:qscores}
\end{longtable}
\end{small}
\end{singlespacing}
\end{landscape}
%\end{minipage}

%\textbf{Reproducibility:} A basic metric that indicates differences or similarities for read densities of different ChIP-seq experiments is the correlation coefficient \citep{Bardet2012} which can, for example, be used to compare replicates (two replicates are recommended for ChIP-seq experiments \citep{Rozowsky2009, Landt2012}). However, particularly the Pearson correlation coefficient is sensitive to outliers and could, for example, be skewed by artificially overrepresented loci such as subtelomeric and Ultra High Signal \citep{Bailey2013, Blacklists} (see \tref{tab:biases}). 

%\textbf{ChIP strength:} In the past two years, several measures for signal-to-noise ratios have been proposed such as the normalized and relative cross-correlation coefficients \citep{Landt2012} and the visualization of cumulative percentages of read counts \citep{Diaz2012}. Both measures work well for highly localized enrichments, but are much less informative for ChIP-seq experiments for which broad, domain-like enrichments are expected (e.g. H3K27me3).
