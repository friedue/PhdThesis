%\begin{minipage}{\textwidth} % to avoid pagebreaks despite longtable environment
\begin{singlespacing}
\begin{small}
\setlength{\extrarowheight}{2pt}
%\vspace*{-2em}
\begin{longtable}[l]{>{\textsf\bgroup}p{5.5cm}<{\egroup} >{\textsf\bgroup}p{8.5cm}<{\egroup}} % defining the columns - these must match the widths defined for the mini pages down below!
\caption[Protein names of MSL- and NSL complex members in \textit{D.~melanogaster} and mammals.]{\textsf{Protein names of MSL- and NSL complex members in \textit{D.~melanogaster}, mouse and humans. Bold font indicates the name that I used throughout most of the manuscript unless noted otherwise. MSL = male-specific lethal, NSL = non-specific lethal}} \\ % the \\ is important!
%%%%%%%%%%%%%%%
% table title
\textbf{\textit{Drosophila}} & \textbf{Mammals}
\tabularnewline \hline
\endfirsthead 
%
\multicolumn{2}{c}%
{\tablename\ \thetable\ -- \textit{Continued from previous page}} \\[1ex]
\textbf{\textit{Drosophila}} & \textbf{Mammals}
\tabularnewline \toprule %\tabularnewline [1ex]
\endhead % Line(s) to appear at top of every page (except first)
\multicolumn{2}{r}{\textit{Continued on next page}} \\
\endfoot % Last line(s) to appear at the bottom of every page (except last)
\endlastfoot
%-----------------------
%%%%%%%%%%%%%%%
 \textbf{MOF} (males absent on first) & MYST1, hMOF, KAT8 (lysine acetyl transferase 8)
\tabularnewline \midrule
%--------------------------
\textbf{MSL1} & hMSL1
\tabularnewline \midrule
%--------------------------
\textbf{MSL2} & hMSL2, Msl2l1, Rnf184 (ring finger~184)
\tabularnewline \midrule
%--------------------------
\textbf{MSL3} & hMSL3, Msl31
\tabularnewline \midrule
%--------------------------
\textbf{MLE} (maleless) & DHX9 (aspartic-acid-glutamine-alanine-histidin (DEAH) box helicase~9)
\tabularnewline \midrule
%--------------------------
\textbf{NSL1}, waharan & hMSL1v1, Kansl1 (KAT8 regulatory NSL complex subunit~1)
\tabularnewline \midrule
%--------------------------
\textbf{NSL2}, dgt1~(dimmed gamma tubulin~1) & Kansl2 (KAT8 regulatory NSL complex subunit~2)
\tabularnewline \midrule
%--------------------------
\textbf{NSL3}, Rcd1~(reduction in cen\-tro\-somin dots) & Kansl3 (KAT8 regulatory NSL complex subunit~3)
\tabularnewline \midrule
%--------------------------
\textbf{MBD-R2} (methyl-binding domain protein) & PHF20~(plant homeo domain~(PHD) finger protein~20), GLEA2~(glioma-expressed antigen~2)
\tabularnewline \midrule
%--------------------------
\textbf{MCRS2}, dMCRS1, p78, Rcd5~(reduction in cen\-tro\-somin dots) & p78, MSP58~(microspherule protein~58)
\tabularnewline \midrule
%--------------------------
\textbf{WDS} (will die slowly) & WDR5 (tryptophan-aspartic acid~(WD) repeat domain~5)
%--------------------------
\tabularnewline \bottomrule
%-----------------------------
\label{tab:Names}
\end{longtable}
\end{small}
\end{singlespacing}
%\end{minipage}