%\begin{minipage}{\textwidth} % to avoid pagebreaks despite longtable environment
%\setlength{\extrarowheight}{2pt}
%\renewcommand{\arraystretch}{1.5}
\vspace*{-2em}
%\begin{longtable}[l]{>{\textsf\bgroup}p{3cm}<{\egroup} >{\textsf\bgroup}p{2.5cm}<{\egroup}>{\textsf\bgroup}p{5cm}<{\egroup}} % defining the columns - these must match the widths defined for the mini pages down below!
\begin{table}[t]
\begin{singlespacing}
\begin{small}
\begin{sffamily}
\caption[The main histone acetyl transferase families.]{\textsf{The main histone acetyl transferase families. The lysine acetyl transferase (KAT) designation indicates the standardized nomenclature suggested for HATs. GNAT = Gcn5-related \textsl{N}-acetyl transferase, MYST = MOZ, Ybf2/Sas3, Sas2, Tip60. The table was taken from \citet{Berndsen2008}}} %\\ % the \\ is important!
\label{tab:HATs}
\begin{tabular}{>{\small\textsf\bgroup}p{3.5cm}<{\egroup} >{\small\textsf\bgroup}p{3.5cm}<{\egroup} >{\small\textsf\bgroup}p{5cm}<{\egroup} }
%%%%%%%%%%%%%%%
% table title
\textbf{Enzyme} & \textbf{KAT designation} & \textbf{Histone specificity}
\tabularnewline \toprule
%-----------------------
\textbf{GNAT family} & & \\
Gcn5 & 2 & H3K9, 14, 36 \\
p/CAF & 2B & H3K14
\tabularnewline \midrule
%--------------------------
\textbf{MYST family} & & \\
Tip60 & 5 & H4K5, K8, K12 (K16) \\
MOF 	&	8	&	H4K5, K8, K16 \\
Sas3	&	6	&	H3K14, K23 \\
MOZ		&	6A & H3K14
\tabularnewline \midrule
%--------------------------
\textbf{p300 and others} & & \\
CBP & 3A & H2AK5, H2B \\
p300 	&	3B	&	H2AK5, H2B \\
Rtt109	&	11 & H3K56, K9, K23	
%--------------------------
\tabularnewline \bottomrule
%-----------------------------
\end{tabular}
%\label{tab:HATs}
%\end{longtable}
\end{sffamily}
\end{small}
\end{singlespacing}
\end{table}
%\end{minipage}
