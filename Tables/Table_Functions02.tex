%\vspace*{-2em}
\begin{minipage}{\textwidth}
\begin{singlespacing}
\begin{small}
\begin{sffamily}
\begin{longtable}[l]{>{\raggedright\arraybackslash}p{3cm} >{\raggedright\arraybackslash}p{11cm}}
\caption[MOF-associated proteins in cell-cycle-related processes.]{\textsf{MOF-associated proteins in cell-cycle-related processes. Related to \fref{fig:functions}}} \\ % the \\ is important! 
%%%%%%%%%%%%%%%
% table title
\textbf{Biological process} & \textbf{Observation} 
\tabularnewline \toprule
%==========================
\begin{minipage}[c]{3cm}
					G2/M checkpoint
			\end{minipage}
					& \begin{minipage}[c]{11cm} % 3rd column
					\vskip 4pt
							NSL1, NSL2, NSL3, MCRS2, MBD-R2 and WDS were identified as essential factors for G2/M checkpoint progression following DNA damage in \textit{D.~melano\-gaster} \citep{Kondo2011}
							\vskip 4pt
							\end{minipage}
\tabularnewline \midrule
%\\ [2ex] \hline \\ [1ex]
%-----------------------
\begin{minipage}[c]{3cm}
					mitotic spindle
			\end{minipage}
					& \begin{minipage}[c]{11cm} % 3rd column
					\vskip 4pt
							\begin{itemize}[noitemsep, leftmargin=*]
							\item NSL2 is needed for mitotic spindle assembly \citep{Goshima2007}
							\item MCRS1 stabilizes the mitotic spindle \citep{Meunier2011}
							\item WDS was identified in a screen for microtubule-associated proteins \cite{Hughes2008}
							\end{itemize}
							\vskip 2pt
							\end{minipage}
\tabularnewline \midrule
%------------------------
\begin{minipage}[c]{3cm}
					apoptosis
			\end{minipage}
					& \begin{minipage}[c]{11cm} % 3rd column
					\vskip 4pt
							\begin{itemize}[noitemsep, leftmargin=*]
							\item ubiquitylation of p53 by MSL2 leads to accumulation of p53 in the cytoplasm \citep{Kruse2009} which is necessary for apoptosis \cite{Muscolini2011}
							\item MOF acetylates p53 in the presence of NSL1 \citep{Li2009} which is necessary for apoptosis induction in cells with DNA damages \citep{Sykes2006, Sykes2009}
							\item human MBD-R2 (PHF20) stimulates expression of p53 and prevents its degradation via a direct interaction with methylated p53 \citep{Park2009, Cui2012}
							\end{itemize}
							\vskip 2pt
							\end{minipage}
\tabularnewline \midrule
%------------------------
\begin{minipage}[c]{3cm}
					DNA repair
			\end{minipage}
					& \begin{minipage}[c]{11cm} % 3rd column
					\vskip 4pt
							\begin{itemize}[noitemsep, leftmargin=*]
								\item MOF is generally required for repair of DNA double strand breaks and recruitment of 53BP1 and BRCA \cite{Sharma2010, Li2010}; its phosphorylated form is particularly important for biasing the cells towards homologous repair during S~phase by displacing 53BP1 from the site of the DNA damage \citep{Gupta2014}
								\item human MSL2 ubiquitylates 53BP1 \citep{Lai2013}
								\item human MSL1 interacts with 53BP1 that positively stimulates DNA damage repair \citep{Gironella2009}
							\end{itemize}
							\vskip 2pt
							\end{minipage}
%%%%%%%%%%%%%%%%%%%%%%%%%%
\tabularnewline \bottomrule
\label{tab:functions2}
\end{longtable}
\end{sffamily}
\end{small}
\end{singlespacing}
\end{minipage}
