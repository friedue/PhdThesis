% Chapter Template

%\chapter{Aims} % Main chapter title

%\label{Aims} % for referencing this chapter elsewhere, use \ref{ChapterX}

%\lhead{Aims} % this is for the header on each page - perhaps a shortened title

%------------------------------------------------------------------
%	SECTION 
The goal of my studies was to gain biological insights from the analyses of the binding profiles of MOF-associated proteins: the non-specific lethal complex (NSL) and the male-specific lethal complex (MSL) members.
The analyses were based on chromatin immunoprecipitation followed by high-throughput sequencing (ChIP-seq). Thus, before the biological questions could be addressed, we first needed to establish robust bioinformatic workflows and scripts ranging from data processing to bias identification and correction and manifold customized downstream analyses. 

In \textit{Drosophila}, we specifically wanted to address whether the different complex members of the NSL complex would co-occur. Moreover, we wanted to infer and subsequently test new hypotheses about the biological function of the NSL complex in regard to gene expression regulation.
Since both MSL and NSL complex had not been previously examined in mammals, we then set out to study their chromatin targeting in mouse cells. To this end, Tomasz Chelmicki generated ChIP-seq profiles of MOF, MSL1, MSL2, NSL3 and MCRS1 in mouse embryonic stem cells and neuronal progenitor cells. The ChIP-seq data study revealed common and different binding principles of the two complexes in pluripotent and differentiated cells which we complemented with transcriptome studies from perturbation experiments and additional genome-wide data sets.



